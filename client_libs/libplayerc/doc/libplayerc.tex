\documentclass[11pt]{report}
\usepackage{fullpage}

\def\VERSION {1.2}
\def\DEFAULTPORT {6665}
\def\HOMEPAGE {{\tt http://playerstage.sourceforge.net}}
\def\libplayerc {{\tt libplayerc} }

\begin{document}
\setcounter{page}{0}
\pagenumbering{roman}

\titlepage

\begin{flushright}
\begin{tabular}{r}
{\bf USC Robotics Laboratory}\\
University of Southern California\\
Los Angeles, California, USA\\
\end{tabular}
\end{flushright}

\vspace{5cm}
\centerline{\huge{libplayerc}}
\vspace{0.5cm}
\centerline{\large{Version \VERSION\ Reference Manual}}
\vspace{2cm}

\centerline{\large Andrew Howard}
\centerline{\sl ahoward@usc.edu}
\vspace{5cm}
\centerline{\today}

\newpage
\tableofcontents

\newpage
\setcounter{page}{0}
\pagenumbering{arabic}

\chapter{Introduction}

\section{Licence}

\section{Bugs}

\section{Acknowledgements}

\chapter{General Usage}

Like the C++ client, \libplayerc is based on a ``proxy'' model in
which the client maintains a local proxy for each of the remote
devices on the server.  Thus, for example, one can create proxies for
the {\tt position} and {\tt laser} devices.  There is also a special
{\tt client} proxy, used to interact with the server itself.  Programs
using \libplayerc will generally have the following structure.

\begin{quote}
\scriptsize
\begin{verbatim}
#include <stdio.h>
#include "playerc.h"

int main(int argc, const char **argv)
{
  int i;
  playerc_client_t *client;
  playerc_position_t *position;

  client = playerc_client_create(NULL, "localhost", 6665);
  if (playerc_client_connect(client) != 0)
    return -1;

  position = playerc_position_create(client, 0);
  if (playerc_position_subscribe(position, PLAYER_ALL_MODE))
    return -1;
  if (playerc_position_enable(position, 1) != 0)
    return -1;
  if (playerc_position_setspeed(position, 0, 0, 0.1) != 0)
    return -1;
  
  for (i = 0; i < 200; i++)
  {
    playerc_client_read(client);
    printf("position : %f %f %f\n",
           position->px, position->py, position->pa);
  } 

  playerc_position_unsubscribe(position);
  playerc_position_destroy(position);
  playerc_client_disconnect(client);
  playerc_client_destroy(client);

  return 0;
}
\end{verbatim}
\end{quote}
BEWARE: this program will cause the robot to spin in circles!

\begin{itemize}
\item The client proxy is created using:
\begin{quote}\begin{verbatim}
client = playerc_client_create(NULL, "localhost", 6665);
\end{verbatim}\end{quote}
where \verb+"localhost"+ is the host name 

\item 
\begin{quote}\begin{verbatim}
playerc_client_connect(client)
\end{verbatim}\end{quote}

\end{itemize}


\chapter{Device Proxy Reference}

\section{{\tt playerc\_client\_t} : the client proxy}




\section{{\tt playerc\_position\_t} : the position device proxy}




\appendix

\chapter{Source Listing : \tt playerc.h}

\end{document}


