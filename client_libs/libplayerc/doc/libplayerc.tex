\documentclass[11pt]{report}
\usepackage{fullpage}
\usepackage{listings}
\lstset{language=C, basicstyle=\footnotesize, frame=tb}

\def\VERSION {1.2}
\def\DEFAULTPORT {6665}
\def\HOMEPAGE {{\tt http://playerstage.sourceforge.net}}
\def\libplayerc {{\tt libplayerc} }

\begin{document}
\setcounter{page}{0}
\pagenumbering{roman}

\titlepage

\begin{flushright}
\begin{tabular}{r}
{\bf USC Robotics Laboratory}\\
University of Southern California\\
Los Angeles, California, USA\\
\end{tabular}
\end{flushright}

\vspace{5cm}
\centerline{\huge{libplayerc}}
\vspace{0.5cm}
\centerline{\large{Version \VERSION\ Reference Manual}}
\vspace{2cm}

\centerline{\large Andrew Howard}
\centerline{\sl ahoward@usc.edu}
\vspace{5cm}
\centerline{\today}

\newpage
\tableofcontents

\newpage
\setcounter{page}{0}
\pagenumbering{arabic}

\chapter{Introduction}

\libplayerc is a client library for the Player robot device server.
It is written in ANSI-C to maximize portability.

Users should read the {\em Player User Manual} included in the standard
Player distribution.  

The latest version  manual can also be found on the

\section{Getting \libplayerc}

\libplayerc is included with the standard Player distribution, starting with
version 1.2.  The Player homepage is:\\
\indent \HOMEPAGE\\
Check there for the latest versions of the server and this
document.

\section{Bugs}

This software is provided WITHOUT WARRANTY.  Nevertheless, if you find
something that doesn't work, or there is some feature you would like
to see, you can submit a bug report/feature request through the
Player/Stage homepage:\\ \indent \HOMEPAGE\\ Include a detailed
description of you problem and/or feature request, and information
such as the Player version and operating system.  Make sure you also
select the ``\libplayerc'' category when reporting bugs.

\section{Licence}

This program is free software; you can redistribute it and/or modify
it under the terms of the GNU General Public License as published by
the Free Software Foundation; either version 2 of the License, or (at
your option) any later version.  This program is distributed in the
hope that it will be useful, but WITHOUT ANY WARRANTY; without even
the implied warranty of MERCHANTABILITY or FITNESS FOR A PARTICULAR
PURPOSE.  See the GNU General Public License for more details.  You
should have received a copy of the GNU General Public License along
with this program; if not, write to the Free Software Foundation,
Inc., 59 Temple Place - Suite 330, Boston, MA 02111-1307, USA.

\section{Acknowledgements}

This work is supported by DARPA grant DABT63-99-1-0015 (MARS) and NSF
grant ANI-9979457 (SCOWR), and possibly others.


\chapter{General Usage}

\libplayerc is based on a device ``proxy'' model, in which the client
maintains a local proxy for each of the devices on the remote server.
Thus, for example, one can create local proxies for the {\tt position}
and {\tt laser} devices.  There is also a special {\tt client} proxy,
used to control the Player server itself.  

Programs using \libplayerc will generally the following structure:
\begin{quote}
\begin{lstlisting}[frame=tb]{}
#include <stdio.h>
#include "playerc.h"

int main(int argc, const char **argv)
{
  int i;
  playerc_client_t *client;
  playerc_position_t *position;

  client = playerc_client_create(NULL, "localhost", 6665);
  playerc_client_connect(client);

  position = playerc_position_create(client, 0);
  playerc_position_subscribe(position, PLAYER_ALL_MODE);

  playerc_position_enable(position, 1);
  playerc_position_setspeed(position, 0, 0, 0.1);
  
  for (i = 0; i < 200; i++)
  {
    playerc_client_read(client);
    printf("position : %f %f %f\n",
           position->px, position->py, position->pa);
  } 

  playerc_position_unsubscribe(position);
  playerc_position_destroy(position);
  playerc_client_disconnect(client);
  playerc_client_destroy(client);

  return 0;
}
\end{lstlisting}
\end{quote}
Note that error checking has been omitted from this example for the
sake of clarity.  For an example with full error checking, see {\tt
simple.c} in the {\tt examples/libplayerc} directory.  This program
can be broken into six steps, as follows.

\subsection*{Create and connect a client proxy.}

\begin{quote}\begin{verbatim}
client = playerc_client_create(NULL, "localhost", 6665);
playerc_client_connect(client);
\end{verbatim}\end{quote}
The {\tt create} function creates a new client proxy and returns a
pointer to be used in future function calls ({\tt localhost} should be
replaced with the network host name of the robot).  The {\tt connect}
function notifies the Player server that a new client wishes to
recieve data.

\subsection*{Create and subscribe a device proxy.}

\begin{quote}\begin{verbatim}
position = playerc_position_create(client, 0);
playerc_position_subscribe(position, PLAYER_ALL_MODE);
\end{verbatim}\end{quote}
The {\tt create} function creates a new position device proxy and
returns a pointer to be used in future function calls.  The {\tt
subscribe} function notifies the Player server that the client is
using the position device, and that the client expects to both send
commands and recieve data ({\tt PLAYER\_MODE\_ALL}).

\subsection*{Configure the device, send commands.}

\begin{quote}\begin{verbatim}
playerc_position_enable(position, 1);
playerc_position_setspeed(position, 0, 0, 0.1);
\end{verbatim}\end{quote}
The {\tt enable} function sends a configuration request to the server,
changing the robot's motor state from {\tt off} to {\tt on}, thereby
allowing the robot to move.  The {\tt setspeed} function sends a new
motor speed, in this case commanding the robot to turn on the spot.

Note that most Player devices will accept both asynchronous {\em
command} and synchronous {\em configuration} requests.  Sending
commands is analogous using the standard Unix {\tt write} device
interface, while sending configuration requests is analogous to using
the {\tt ioctl} interface.  For the most part, {\tt libplayerc} hides
the distinction between these two interfaces.  Users should be aware,
however, that while commands are always handled promptly by the
server, configuration requests may take significant time to complete.
If possible, configuration requests should therefore be restricted to
the initialization phase of the program.

\subsection*{Read data from the device.}

\begin{quote}\begin{verbatim}
playerc_client_read(client);
printf("position : %f %f %f\n", position->px, ... );
\end{verbatim}\end{quote}
The {\tt read} function blocks until new data arrives from the Player
server.  This data may be from one of the subscribed devices, or it
may be from the server itself (which sends regular synchronization
messages to all of its clients).  The {\tt read} function inspects the
incoming data and automatically updates the elements in the
appropriate device proxy.  This function also returns a pointer to the
proxy that was updated, so that user programs may, if desired, trigger
appropriate events on the arrival of different kinds of data.

\subsection*{Unsubscribe and destroy the device proxy.}

\begin{quote}\begin{verbatim}
playerc_position_unsubscribe(position);
playerc_position_destroy(position);
\end{verbatim}\end{quote}
The {\tt unsubscribe} function tells the Player server that the client
is no longer using this device.  The {\tt destroy} function then frees
the memory associated with the device proxy; the {\tt device} pointer
is now invalid and should be not be re-used.

\subsection*{Disconnect and destroy the client proxy.}

\begin{quote}\begin{verbatim}
playerc_client_disconnect(client);
playerc_client_destroy(client);
\end{verbatim}\end{quote}
The {\tt disconnect} function tells the server that the client is
shutting down.  The {\tt destroy} function then frees the memory
associated with the client proxy; the {\tt client} pointer is now
invalid and should be not be re-used.




\chapter{Device Proxy Reference}

\section{The {\tt laser} proxy}

The {\tt laser} proxy has the following structure.
\begin{quote}
\lstinputlisting{laser.t.tex}
\end{quote}

\noindent The {\tt laser} proxy supports the following functions.
\begin{quote}
\lstinputlisting{laser.f.tex}
\end{quote}


\section{The {\tt position} proxy}

The {\tt position} proxy has the following structure.
\begin{quote}
\lstinputlisting{position.t.tex}
\end{quote}

\noindent The {\tt position} proxy supports the following functions.
\begin{quote}
\lstinputlisting{position.f.tex}
\end{quote}


\end{document}


